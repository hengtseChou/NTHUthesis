\chapter{緒論}
\label{c:intro}

統計學是現代研究的核心工具,無論是在自然科學、社會科學還是商業領域,其應用都展現了極大的價值。隨著數據量的快速增長和計算能力的提高,先進的統計方法和建模技術越來越受到重視。這些技術不僅能夠更準確地描述數據的內在結構,還能為複雜系統中的決策提供強有力的支持,參見 \cite{Alpher02}。

近年來,機器學習與統計建模的融合進一步拓展了統計學的應用邊界。例如,回歸分析和貝葉斯推斷等傳統統計方法與深度學習等技術相結合,顯著提升了預測的準確性和效率。然而,根據 \cite{Alpher03},這些技術的應用也帶來了一些挑戰,例如模型假設的合理性、結果的可解釋性以及大規模數據集的處理等問題。

本論文旨在探討先進統計建模技術在解釋與預測中的應用,並針對大數據環境下的挑戰提出解決方案。本研究的主要貢獻在於開發和評估一套綜合性統計框架,用於增強模型的適應性和穩健性,並提供可行的政策建議,幫助研究者和實務者在各自領域中實現最佳決策。
