\chapter{研究方法}
\label{c:methods}
本章介紹本研究中採用的研究方法,包括數據處理、模型構建和實驗設計。為了更清晰地描述方法,我們提供了一個數據表、一個算法流程以及一個圖形示例。

\section{數據處理}

數據處理是本研究的重要步驟,主要包括數據清理、標準化和特徵提取。以下是本研究使用的數據摘要表:

\begin{table}[htbp]
\centering
\begin{tabularx}{\textwidth}{@{}lXX@{}}
\toprule
\textbf{Feature Name} & \textbf{Type} & \textbf{Description} \\ 
\midrule
Age                   & Numerical      & Participant's age (in years). \\ 
Gender                & Categorical    & Gender of the participant (Male/Female). \\ 
Income                & Numerical      & Annual income (in USD). \\ 
Purchase Intent       & Categorical    & Whether there is purchase intent (Yes/No). \\ 
\bottomrule
\end{tabularx}
\caption{Summary of dataset features}
\label{tab:data_summary}
\end{table}

\section{模型構建}

模型構建基於先進的機器學習算法。以下展示了一個通用的模型訓練過程的算法描述:

\begin{algorithm}[htbp]
\caption{模型訓練算法}
\label{alg:model_training}
\KwIn{Dataset \( D = \{(x_i, y_i)\}_{i=1}^n \), learning rate \( \eta \), number of iterations \( T \)}
\KwOut{Trained model parameters \( \theta \)}

Initialize model parameters \( \theta \)\;
\For{\( t = 1 \) \KwTo \( T \)}{
    \For{\( (x_i, y_i) \in D \)}{
        Compute prediction \( \hat{y}_i = f(x_i; \theta) \)\;
        Update parameters \( \theta \leftarrow \theta - \eta \cdot \nabla_\theta \mathcal{L}(\hat{y}_i, y_i) \)\;
    }
}
\Return \( \theta \;
\)
\end{algorithm}

\section{實驗設計}

為了驗證模型的性能,我們設計了一個簡單的實驗,並使用數據可視化來呈現結果。圖 \ref{fig:example_figure} 展示了一個假設的數據分佈圖。

\begin{figure}[htbp]
\centering
\includegraphics[width=0.4\textwidth]{example-image-a}
\caption{數據分佈圖示例}
\label{fig:example_figure}
\end{figure}

本章詳細介紹了數據處理、模型構建和實驗設計的過程。後續章節將展示這些方法的應用結果及其對研究問題的解決。
