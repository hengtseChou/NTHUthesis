\chapter{結論}
\label{c:conclusions}

本研究提出了一種基於統計建模的框架,並探索了其在大數據環境中的應用價值。在本論文中,我們通過數據處理、模型構建以及實驗分析的方法,驗證了所提出方法的有效性與可行性。以下是本研究的主要結論:

本研究的主要貢獻如下:
\begin{enumerate}
  \item \textbf{理論貢獻}:本研究進一步拓展了統計建模在大數據環境下的應用,並提出了一套改進的算法流程,有助於提高模型的穩健性與準確性。
  \item \textbf{實踐貢獻}:通過應用於實際數據集,本研究提供了一套可實現的工具,幫助從業者進行更準確的數據分析與預測。
  \item \textbf{方法創新}:本研究在傳統統計理論的基礎上,結合了機器學習技術,有效解決了模型假設和數據異質性帶來的挑戰。
\end{enumerate}

總之,本研究展示了統計建模在大數據分析中的廣泛潛力,並為後續研究提供了堅實的基礎與方向。期待未來的研究能進一步解決現有挑戰,並推動相關領域的進一步發展。
