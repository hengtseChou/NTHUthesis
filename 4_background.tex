\chapter{研究背景}
\label{c:background}

在本章中,我們將介紹本研究的背景,並解釋其與現代統計學研究的關聯性。本章包括一個主要部分,其中進一步劃分為一個小節,涵蓋本研究的理論基礎和應用範圍。

\section{理論基礎}

統計學的發展為研究複雜系統提供了豐富的工具,其中數學理論和實證方法的結合是其成功的關鍵。在本節中,我們將討論本研究所依賴的一些基本理論概念和相關應用。

\subsection{定義與基本性質}

為了說明本研究所涉及的理論框架,我們首先引入一些基本的定義和相關的理論結論。

\begin{definition}[隨機變量]
設 \( X \) 為一個定義在樣本空間 \( \Omega \) 上的函數。如果對於任何實數 \( x \),集合 \( \{\omega \in \Omega \mid X(\omega) \leq x\} \) 為事件,則稱 \( X \) 為隨機變量。
\end{definition}

\begin{example}
設 \( \Omega = \{1, 2, 3, 4, 5, 6\} \) 表示擲一顆公平骰子的樣本空間。如果定義 \( X(\omega) = \omega \),則 \( X \) 為隨機變量,並且其可能取值為 \( \{1, 2, 3, 4, 5, 6\} \)。
\end{example}

\begin{proposition}
設 \( X \) 和 \( Y \) 為獨立的隨機變量,則其和 \( Z = X + Y \) 的期望值為
\[
E[Z] = E[X] + E[Y].
\]
\end{proposition}

\begin{proof}
由期望的線性性質,我們有:
\[
E[Z] = E[X + Y].
\]
由於 \( X \) 和 \( Y \) 為獨立隨機變量,我們可以將它們的期望分開:
\[
E[X + Y] = E[X] + E[Y].
\]
因此,結論成立,即:
\[
E[Z] = E[X] + E[Y].
\]
\end{proof}

\begin{lemma}
若隨機變量 \( X \) 服從正態分佈 \( N(\mu, \sigma^2) \),則對於任何常數 \( a \) 和 \( b \),隨機變量 \( Y = aX + b \) 服從正態分佈 \( N(a\mu + b, a^2\sigma^2) \)。
\end{lemma}

\begin{theorem}
設 \( X_1, X_2, \dots, X_n \) 為互相獨立且具有相同分佈的隨機變量,且 \( E[X_i] = \mu \),\( \text{Var}(X_i) = \sigma^2 \)。則當 \( n \to \infty \) 時,其樣本均值
\[
\bar{X}_n = \frac{1}{n} \sum_{i=1}^n X_i
\]
服從正態分佈,且有
\[
\bar{X}_n \sim N\left(\mu, \frac{\sigma^2}{n}\right).
\]
\end{theorem}

\begin{corollary}
由定理 1.1 推得,當樣本數量足夠大時,任何分佈的樣本均值都可近似服從正態分佈。此性質對於統計推斷和參數估計具有重要意義。
\end{corollary}

本節討論了本研究的理論基礎,並提供了相關的定義、例子和理論結果。在後續章節中,我們將基於這些理論框架展開進一步的應用和分析。
